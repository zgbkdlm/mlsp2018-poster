\documentclass[portrait,a1,final]{a0poster} % a0poster class sets the paper size
\usepackage[utf8]{inputenc}
\usepackage[english]{babel}
\usepackage[pdftex]{graphicx}
\usepackage[ELEC,RGB]{aaltologo} % RGB, Coated, Uncoated
\usepackage{lipsum} % Lorem ipsum generator

\usepackage{titlesec} % For changing the font on chapters, sections, etc.


%% Section font size reduction for a1 posters %%%%%%%%%%%%%%%%%%%%%%%%%%%%%%%%%%%%%%%%%

% Also changes the format to sans serif bold, i.e. bold Helvetica via aaltologo package, can change the color of section text

%\titleformat{\section}{\large\bfseries\sffamily\color{aaltoFuchsia}}{\textcolor{aaltoFuchsia}{\thesection}}{1em}{} % Text size for a1 posters

\titleformat{\section}{\large\bfseries\sffamily\color{aaltoBlack}}{\textcolor{aaltoBlack}{\thesection}}{1em}{} % Text size for a1 posters

%\titleformat{\section}{\large\bfseries\sffamily}{\thesection.}{1em}{} % Text size for a1 posters with a dot after the incremental number

%%%%%%%%%%%%%%%%%%%%%%%%%%%%%%%%%%%%%%%%%%%%%%%%%%%%%%%%%%%%%%%%%%%%%%%%%%%%%%%%%%%%%%%


\usepackage{epstopdf} % so that ``pdflatex'' accepts .eps files


% Safe alternative math font for aaltoseries (optional)
\usepackage{fouriernc}


% Teach hyphenation for alien words
\hyphenation{op-tical net-works semi-conduc-tor}






%\newcommand{\vt}[1]{\ensuremath{\boldsymbol{#1}}} % vector, i.e. math bold face, cursive
\newcommand{\vt}[1]{\ensuremath{\mathbf{#1}}} % vector, i.e. math bold face, not-cursive
\newcommand{\lt}[1]{\ensuremath{\mathrm{#1}}} % sub or sup text, i.e. not cursive
\newcommand{\ec}{\ensuremath{\vt{J}}} % electric surface current
\newcommand{\mc}{\ensuremath{\vt{M}}} % magnetic surface current
%
\newcommand{\aee}{\ensuremath{\overline{\overline{a}}_\lt{ee}}}
\newcommand{\aem}{\ensuremath{\overline{\overline{a}}_\lt{em}}}
\newcommand{\ame}{\ensuremath{\overline{\overline{a}}_\lt{me}}}
\newcommand{\amm}{\ensuremath{\overline{\overline{a}}_\lt{mm}}}
\newcommand{\bee}{\ensuremath{\overline{\overline{b}}_\lt{ee}}}
\newcommand{\bem}{\ensuremath{\overline{\overline{b}}_\lt{em}}}
\newcommand{\bme}{\ensuremath{\overline{\overline{b}}_\lt{me}}}
\newcommand{\bmm}{\ensuremath{\overline{\overline{b}}_\lt{mm}}}
%
\newcommand{\adyad}{\ensuremath{\overline{\overline{a}}}}
\newcommand{\bdyad}{\ensuremath{\overline{\overline{b}}}}
%
\newcommand{\unitx}{\ensuremath{\vt{x}_0}}
\newcommand{\unity}{\ensuremath{\vt{y}_0}}






\newcommand{\sectionspace}{10mm} % Free space before each section inside a minipage
\newcommand{\figurespace}{10mm} % Free space around figures inside a minipage (where floats are not allowed)


\begin{document}






\thispagestyle{empty} % Removes the page number




\begin{minipage}[t]{0.98\linewidth} % The first minipage for the logo & title
\vspace{0pt} % A trick to align the parallel minipages on top

\vspace{0.008\linewidth} % Increase the top margin

\begin{minipage}[t]{0.28\linewidth} % logo
\vspace{0pt} % Alingns the parallel minipages on top

%% Choose the logo or use random generator
%\AaltoLogoLarge{1.55}{''}{aaltoBlue} % Chosen logo, scaled for A1 size
\AaltoLogoRandomLarge{1.55} % Random logo, scaled for A1 size

\end{minipage} % no empty line before the next begin
\begin{minipage}[t]{0.7\linewidth} % title
\vspace{0pt} % Alingns the parallel minipages on top


%% Font sizes for a0poster are
%\tiny
%\scriptsize
%\footnotesize
%\small
%\normalsize
%\large
%\Large
%\LARGE
%\huge
%\Huge
%\veryHuge
%\VeryHuge
%\VERYHuge

%% Official colors from aaltologo-package (visual-identity guideline)
% aaltoBlack
% aaltoGray
% aaltoGrayScale (for b&w prints)
% aaltoYellow
% aaltoOrange
% aaltoRed
% aaltoFuchsia
% aaltoPurple
% aaltoBlue
% aaltoTurquoise
% aaltoGreen
% aaltoLightGreen
% You can use \textcolor{<aaltocolor>}{<your text>) to change the colors of text, or

%% Aalto-fancy title, use \baselinestretch to change linespacing, \textit{} for italic text, \textcolor for colored text
%{\renewcommand{\baselinestretch}{0.6} % Changes the baseline skip smaller for the title
%\textcolor{aaltoGreen}{\veryHuge{\bfseries{\textsf{The title of the poster\\ that can} \textit{span\\ to multiple rows}}}} % Text size for a1 posters
%\par} % <- for \baselinestretch

% More conservative title, upright and black
{\renewcommand{\baselinestretch}{0.85} % Changes the baseline skip smaller for the title
\Huge{\bfseries{\textsf{The title of the poster\\ that can span\\ to multiple rows}}} % Text size for a1 posters
\par} % <- for \baselinestretch


\vspace{0.04\linewidth} % Empty space after the title

\normalsize{\textsf{\bfseries{Roger R.\ Researcher and Paula P.\ Professor}}} % Text size for a1 posters

\textcolor{aaltoGray}{\textsf{\bfseries{Department of Radio Science and Engineering, Aalto University}}}
%
\end{minipage}
\end{minipage}






%% Two columns

% Space according to the visual identity guidelines...
\vspace{0.08\linewidth}

% Centering helps in placement
\centering



\small % Text size for a1 posters


\begin{minipage}{0.98\linewidth}




\begin{minipage}[t]{0.47\linewidth}
\setlength{\parindent}{10mm} % Paragraph indent




\section{Introduction}

\lipsum[1]


\vspace{\sectionspace}
\section{Section}

\lipsum[2]


\vspace{\figurespace}
\begin{center}
  \includegraphics[width=0.45\linewidth]{./figs/chiral_particles_v2.eps}
\end{center}
\vspace{\figurespace}




\begin{itemize}
	\item Each particle can have ten scattering components, but by neglecting the loop polarisability, only four are left \cite{Tretyakov1996}
	\setlength\arraycolsep{2pt}
	\begin{eqnarray}
		\vt{p} & = & (a_\lt{ee}^{(xx)} \unitx\unitx + b_\lt{ee}^{(yy)} \unity\unity) \cdot \vt{E} + (a_\lt{em}^{(xx)} \unitx\unitx + b_\lt{em}^{(yy)} \unity\unity) \cdot \vt{H},
		\label{eq:psimple} \\
		\vt{m} & = & (a_\lt{me}^{(xx)} \unitx\unitx + b_\lt{me}^{(yy)} \unity\unity) \cdot \vt{E} + (a_\lt{mm}^{(xx)} \unitx\unitx + b_\lt{mm}^{(yy)} \unity\unity) \cdot \vt{H},
		\label{eq:msimple}
	\end{eqnarray}
\end{itemize}




\vspace{\sectionspace}
\section{Another section}


\begin{itemize}
		
	\item Figures below
	
\end{itemize}


\vspace{\figurespace}
\begin{center}
  	\includegraphics[width=0.4\linewidth,trim=20mm 10mm 10mm 10mm, clip]{./figs/dir_sca_db_phi00.eps}
  	\includegraphics[width=0.4\linewidth,trim=20mm 10mm 10mm 10mm, clip]{./figs/dir_sca_db_phi90.eps}
\end{center}
\vspace{\figurespace}






% The end of the first column and the start of the second
\end{minipage} % no empty line before the next ``\begin''
\hspace{0.03\linewidth} % Middle margin
\begin{minipage}[t]{0.47\linewidth}
\setlength{\parindent}{10mm} % Paragraph indent








\section{Still one section}

\lipsum[3-4]







\vspace{\sectionspace}
\section{Conclusion}

\begin{itemize}
	\item \lipsum[5]
	\item \lipsum[6]
\end{itemize}



%{\tiny % % Bibliography text size begins...
{\footnotesize % % Bibliography text size begins...

\begin{thebibliography}{1}


\bibitem{Tretyakov1996}
S.~A. Tretyakov, F.~Mariotte, C.~R. Simovski, T.~G. Kharina, and J.~P. Heliot,
  ``Analytical antenna model for chiral scatterers: comparison with numerical
  and experimental data,'' \emph{{IEEE} Trans. Antennas Propag.}, vol.~44,
  no.~7, pp. 1006--1014, 1996.
  
  

\end{thebibliography}

} % <- bibliography text size ends



\end{minipage}
\end{minipage} % minipage for the two columns (minipages)





\vfill % Fill the free space until the footer minipages


\begin{minipage}{0.95\linewidth} % Minipages for the footers


\footnotesize % Text size for footers


\begin{minipage}[t]{0.47\linewidth}% Footer #1
\vspace{0pt}

\textsf{\textbf{Department of Radio Science and Engineering / SMARAD Centre of Excellence\\
School of Electrical Engineering\\
Aalto University, Finland}}

\end{minipage} % No empty line before the second begin!
\hspace{0.03\linewidth}
\begin{minipage}[t]{0.47\linewidth} % Footer #2
\vspace{0pt}

\textsf{\textbf{Contact information for comments \& improvement ideas: Antti Karilainen\\
Email: antti.karilainen@aalto.fi\\
Version 0.1g}}
\end{minipage}



\end{minipage}




\vspace*{0.03\linewidth} % Increase the bottom margins



\end{document}
